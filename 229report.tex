% Latex format built from IEEE Computer Society journal template
% See https://www.ieee.org/conferences_events/conferences/publishing/templates.html
% For full documentation
\documentclass[10pt,journal,compsoc]{IEEEtran}
% *** CITATION PACKAGES ***
\usepackage{listings}
\usepackage{cite}

\newcommand\MYhyperrefoptions{bookmarks=true,bookmarksnumbered=true,
pdfpagemode={UseOutlines},plainpages=false,pdfpagelabels=true,
colorlinks=true,linkcolor={black},citecolor={black},urlcolor={black},
pdftitle={Automated Playlist Generation},
pdfsubject={Playlist Generation},
pdfauthor={Kade Keith, Demetrios Fassois},
pdfkeywords={Playlist Generation, Song Recommendation, Sentiment Analysis}}

\begin{document}

\title{Automated Playlist Generation\\}

\author{
  Kade~Keith,~\IEEEmembership{Student,~Stanford University - Computer Science Department,}
  Demetrios~Fassois,~\IEEEmembership{Student,~Stanford University - Computer Science Department,}
}

% The paper headers
\markboth{Final Project, CS229: Machine Learning Fall 2017}%
{Evaluating Algorithmic Methods for Playing Chess}

\IEEEtitleabstractindextext{
\begin{abstract}
TODO.
\end{abstract}

\begin{IEEEkeywords}
Playlist Generation,
Song Recommendation,
Sentiment Analysis
\end{IEEEkeywords}}

% make the title area
\maketitle

\section{Motivation}

\IEEEPARstart{W}{ith} the growth of musical streaming services, there are now more songs than ever at music listeners fingertips. Because of this growth, the art of constructing playlists has become increasingly challenging, and discovering new music the in the expanse of choices is a daunting task. For this reason, we seek to build an automatic playlist generator, that can take a few songs as a seed set, and generate a complete playlist for the listener.

\subsection{Goal}
Using popular, human-curated playlists as our training data and test data, our system should construct playlists of similar quality. In particular, we plan to incorporate lyrical analysis in our model, as we believe that lyrical content is a big influence when creating playlists.

\section{Method}
We combine data from a number of sources in our project. The primary source is the Million Song Dataset \cite{msd} (MSD), and the corresponding lyrics dataset, which provided lyrics for roughly a quarter of those songs in a bag-of-words format. In addition to that we use Spotify \cite{spotify} as the source of our playlist data, as well as using their's and Last.fm's \cite{lastfm} song info to augment the data from the MSD.

In total, that gives us the following the following group of attributes:
\\\\
\begin{tabular}{lll}
Feature    & Value(s)       & Source          \\
Year       & TODO-TODO      & MSD, Spotify    \\
Tempo      & TODO-TODO      & MSD             \\
TODO       & TODO           & TODO            \\
TODO       & TODO           & TODO            \\
TODO       & TODO           & TODO            \\
TODO       & TODO           & TODO            \\
\end{tabular}
\\\\
TODO REGARDING TIMBRE, TALK ABOUT HOW TO COMPARE TIMBRES

With these features, we approach the task two ways; first as a graph problem, and second as a classification problem.

\subsection{Graph Problem}
TODO

\subsection{Classification}
TODO

\section{Evaluation Method}
TODO - HOW WE SCORE RESULTS

\section{Preliminary experiments}
TODO

\subsection{Graph Problem}
TODO - KNN WITH DISTANCE FUNCTION

\subsection{Classification}
TODO - REGRESSION. PROLLY WITH SCIKIT LEARN

\section{Results}

\subsection{Graph Problem}
TODO - HOW'D IT DO?

\subsection{Classification}
TODO - HOW'D IT DO?

\section{Next steps}
TODO - WHAT'S NEXT

\subsection{Graph Problem}
TODO - EXPLORING PLAYLIST AS PATH IN GRAPH

\subsection{Classification}
TODO - NOT SURE

\section{Contributions}
TODO - WHO DID WHAT

\begin{thebibliography}{9}

\bibitem{msd}
Million Song Dataset \\
\texttt{https://labrosa.ee.columbia.edu/millionsong/}

\bibitem{spotify}
Spotify \\
\texttt{https://www.spotify.com/}

\bibitem{lastfm}
Last.fm \\
\texttt{https://www.last.fm/}

\end{thebibliography}

\begin{appendices}
\section{EXAMPLE APPENDIX}
The following is the output result of the script that tests for the end-game puzzles, looking for checkmate in 1, 2, or 3 move. \\

\lstset{
basicstyle=\small\ttfamily,
columns=flexible,
breaklines=true
}
\begin{lstlisting}
Random
Elapsed time is 0.0160000324249 seconds.
Baseline
Elapsed time is 0.294000148773 seconds.
minimax1
Elapsed time is 11.1240000725 seconds.
AB1Plus1
Elapsed time is 4.76499986649 seconds.
\end{lstlisting}
\end{appendices}

% that's all folks
\end{document}

    Contact GitHub API Training Shop Blog About

    © 2016 GitHub, Inc. Terms Privacy Security Status Help
